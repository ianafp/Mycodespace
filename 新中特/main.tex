\documentclass{article}
\usepackage{ctex}
\usepackage{authblk} 
\usepackage{indentfirst}
\usepackage{geometry}
\usepackage[framemethod=TikZ]{mdframed}
\usepackage{url}   % 网页链接
\usepackage{subcaption} % 子标题
\usepackage{graphicx}
\usepackage{cite}

% Keywords command
\providecommand{\keywords}[1]
{
  \small	
  \textbf{\textit{Keywords---}} #1
}
\title{习近平新时代中国特色社会主义指导下脱贫攻坚与乡村振兴}
\author{张鑫\\22421264,方正班序号98\\ianafp@zju.edu.cn\\计算机科学与技术学院\\\date{}}
\geometry{a4paper,left=2cm,right=2cm,top=1cm,bottom=1cm}

\begin{document}
\setlength{\parindent}{2em} 
\maketitle
\begin{abstract}
    本文在习近平新时代中国特色社会主义思想指导下,探讨了脱贫攻坚与乡村振兴战略的背景、意义和联系。脱贫攻坚战的胜利后,各地区各部门继续推进脱贫地区的乡村振兴,提升农村居民收入和生活质量。乡村振兴战略的提出,旨在解决农业农村现代化问题,实现全面建设社会主义现代化国家的目标。作者在次基础上思考了从脱贫攻坚到乡村振兴的政策一贯性,思考了这其中的习近平新时代中国特色社会主义理论基础,以及作为农村学生的个人思考和感受,并且提出了自己对于乡村振兴战略的建议。


\end{abstract}

\section{政策背景和意义}
\subsection{脱贫攻坚目标已经完成}
"党的十八大以来,以习近平同志为核心的党中央把脱贫攻坚摆在治国理政的突出位置,把脱贫攻坚作为全面建成小康社会的底线任务,以精准扶贫、精准脱贫为基本方略,组织开展了脱贫攻坚人民战争。经过八年接续奋斗,农村贫困人口全部脱贫,绝对贫困得以消除,区域性整体贫困得到解决,脱贫攻坚战取得全面胜利。打赢脱贫攻坚战后,各地区各部门继续深入贯彻落实党中央、国务院决策部署,巩固拓展脱贫攻坚成果,接续推进脱贫地区乡村振兴,脱贫县农村居民收入较快增长,生活质量继续提高。"\cite{govcn20221011}
自2021年2月25日习近平总书记在全国脱贫攻坚总结表彰大会上庄严宣告,我国脱贫攻坚战取得了全面胜利。然而脱贫攻坚的胜利不是终点,全面小康社会也只是习近平中国特色社会主义历史征程的第一个小目标,完成脱贫攻坚后我们一方面要防止返贫等问题,巩固脱贫成果,\cite{hzau2021}另一方面应当趁热打铁,在脱贫攻坚胜利的基础上,着手建设乡村振兴,进一步提升我国农业农村建设水平,迈向共同富裕。
\subsection{乡村振兴战略的提出}
实施乡村振兴战略是党的十九大作出的重大决策部署,对于全面建成社会主义现代化国家有重大的意义。中共中央国务院提出,农业农村问题是关系国计民生的根本问题,没有农业农村的现代化,就没有国家的现代化。
我国乡村目前面临着发展不平衡不充分的严重问题,在供应方面,既有供大于求问题也有供不应求问题,农产品质量亟待提高,农民适应生产力发展和市场竞争的能力不足,新型职业农民队伍建设亟需加强;农村基础设施和民生领域欠账较多,农村环境和生态问题比较突出,乡村发展整体水平亟待提升;国家支农体系相对薄弱,农村金融改革任务繁重,城乡之间要素合理流动机制亟待健全;农村基层党建存在薄弱环节,乡村治理体系和治理能力亟待强化。实施乡村振兴战略,是解决人民日益增长的美好生活需要和不平衡不充分的发展之间矛盾的必然要求,是实现“两个一百年”奋斗目标的必然要求,是实现全体人民共同富裕的必然要求。\cite{zgrmfy2018}
根据以上问题,党的十九大以习近平新时代中国特色社会主义思想为指导,提出了以下乡村振兴基本原则。
\begin{enumerate}
    \item 坚持党管农村工作。
    \item 坚持农业农村优先发展。
    \item 坚持农民主体地位。
    \item 坚持乡村全面振兴。
    \item 坚持城乡融合发展。
    \item 坚持人与自然和谐共生。
    \item 坚持因地制宜、循序渐进。
\end{enumerate}
按照党的十九大提出的决胜全面建成小康社会、分两个阶段实现第二个百年奋斗目标的战略安排,乡村振兴策略要求到2020年,乡村振兴取得重要进展,制度框架和政策体系基本形成;到2035年,乡村振兴取得决定性进展,农业农村现代化基本实现;到2050年,乡村全面振兴,农业强、农村美、农民富全面实现。
\subsection{脱贫攻坚与乡村振兴的联系}
脱贫攻坚和乡村振兴是中国农村发展中的两个重要战略,它们之间存在着紧密的联系和相互支撑的关系,这两个战略具有目标的连续性,脱贫攻坚战略主要目标是消除绝对贫困,确保到2020年所有农村贫困人口实现脱贫,贫困县全部摘帽。而乡村振兴是在脱贫攻坚目标实现后,进一步推动农村全面发展的战略,旨在实现农业农村现代化,提升农民的生活质量。因而这两个战略是一脉相承的关系,脱贫攻坚为乡村振兴打下了坚实的基础,通过解决贫困问题,为乡村振兴提供了必要的前提条件,乡村振兴则是在脱贫攻坚成果基础上的深化和拓展,通过全面推进农村的经济、文化、社会和生态建设,实现农村的可持续发展,同时也是在本质上杜绝了贫困地区返贫的可能性。
脱贫攻坚和乡村振兴之间的紧密联系源于它们在目标、政策、资源、内生动力、共同富裕、社会稳定、治理体系和人才等多个方面的相互支撑与延续。这种联系不仅有助于实现农村的可持续发展,也为中国的整体社会经济发展提供了重要保障。

\section{理论基础}
\subsection{以人为本}
脱贫攻坚战略和乡村振兴战略体现了以人民为中心的发展思想。习近平新时代中国特色社会主义思想强调以人民为中心的发展,脱贫攻坚和乡村振兴战略都是以提高人民生活水平、实现全体人民共同富裕为目标。这体现了对人民利益的高度重视和对人民福祉的深切关怀。
\subsection{新发展理念}
脱贫攻坚战略和乡村振兴战略体现了新发展理念。新发展理念中的“创新、协调、绿色、开放、共享”五大发展理念为脱贫攻坚和乡村振兴提供了理论指导。通过创新驱动发展,协调区域发展,推动绿色发展,扩大对外开放,实现共享发展,促进农村全面进步。
\subsection{社会主义现代化建设}
实现社会主义现代化是中国特色社会主义的总目标之一。脱贫攻坚和乡村振兴是实现农业农村现代化的关键步骤,是全面建设社会主义现代化国家的重要组成部分。
\subsection{作为一名研究生的思考}
作为一名出身于农村地区的研究生,我对习近平中国特色社会主义指导下的脱贫攻坚和乡村振兴战略有直观的感触。
从具体的实施来讲,首先农村地区的产业结构有巨大的调整。为了解决我国农业供不应求和供过于求同时存在的产业结构性问题,我国政府通过补贴激励的形式调整农村作物类型,推出轮耕休耕等策略提高作物多样性的同时对土地进行了养护。
在生产形式方面,推进自动化,机械化,规模化作业,降低农业生产成本,提高农民收入,这是脱贫攻坚的关键。在不适宜进行规模化作业的地区,鼓励进行经济作物的种植,如果树等高附加值作物,因地制宜的进行农业结构改革。
同时,作物的播种方式,农业机械等生产工具也在革新,农业生产效率持续提高,以及农业银行每年为农民提供农业贷款用于覆盖生产成本,这是很多农民开展规模化种植的启动资金。
另外,农村地区的教育,医疗,交通等条件也在逐渐改善,现在的农村呈现出欣欣向荣的氛围,我个人由衷地感谢习近平中国特色社会主义指导下的脱贫攻坚和乡村振兴战略为我的家乡带来的进步。
同时我个人也在思考乡村振兴策略的未来,乡村振兴战略的目标简言之是农村的现代化和产业化,建设适应于农业农村的产业结构,既能满足农村地区居民的共同富裕需求,也能为国家提供高质量的,供给平衡的农业产品以及旅游业等服务业资源。
我认为达成这一点的关键是进一步推进规模化种植,首先要对耕地进行更全局的规划,在满足耕地总量红线的情况下保留大片耕地进行规模化耕种,对于分散的零星耕地则进行精耕细作的高附加值作物种植,同时对农村进行多层次产业规划,如规划建设农产品加工厂,提高农产品附加价值,如规划建设新农村风貌,开展旅游农家乐等服务业建设。这考验基层干部能否在习近平中国特色社会主义指导下因地制宜地,科学地利用农业资源,也需要我们这样的学生努力学习专业知识,学习习近平中国特色社会主义理论,能够回乡投身于农村建设。
\bibliography{ref}{}
\bibliographystyle{plain}
\end{document}